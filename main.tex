\documentclass[12pt]{article}
\usepackage{amsmath}
\usepackage{graphicx}
\usepackage{hyperref}
\usepackage{pgfgantt}
\usepackage{rotating}
\usepackage{mathtools}
\usepackage{dirtytalk}
\usepackage[utf8]{inputenc}
\usepackage[backend=biber, style=authoryear, sorting=nty]{biblatex}
\renewcommand*{\nameyeardelim}{\addcomma\space}
\addbibresource{bibliography.bib}
\graphicspath{{images/}}

\title{%
  Building a programming language without the dependency on English \\
  \large CMP6102 Individual Project:
    Literature Review }

\author{Pandelis Zembashis | S15101590 \\
Supervisor: Emmmett Cooper}

\date{December 2018}

\begin{document}
\maketitle

\clearpage


\section{Area Of Research}

The fundamental aim of this project is to build an accessible programming language. To do so we will need to explore
the base principals of programming language theory. We will look into some of the existing research in place for what
has been defined as good programming practice. This shall be followed by a review of a subset of languages from various
decades and compare how they have evolved from each other and improved on the ideas of previous generations of languages.

A second goal of the project is to make the language accessible to users who are not native english english speakers and
have little to no programming experience. We will be taking a subjective look at the implementation of the languages and
the symbols they use. We'll also take into account the country of origin of the main developers of the language to see if
this has an effect on the symbols used. From this we may be able to deduce if there are indeed hints of linguistic culture
within the design of languages which strengthens the case for a language which strives to be more generic in its choice
of symbols.


\section{If you See What I Mean}

One of the earliest attempts at a programming language that attempted to make use of natural language
to define more readable programs can be found in the design of ISWIM (If you See What I Mean). This
language was first described in a 1966 edition the Association for Computing Machinery's journal
titled "The Next 700 Programming Languages" \parencite{Landin:1966:NPL:365230.365257}.

In this paper the author describes the mistakes make by languages of the time, such as ALGOL60.
They describe programming languages as a means to express things in terms of other things as well as
existing as defining a basic set of given things that can be used. The author attempts to outline a language
that better defines the given things defined by the language by utilising linguistic structure.

The author makes particular reference to mathematical communication in which he describes how
naturally a \emph{where} clause reads.


\begin{gather*}
f(bA - 2c) + f(2b - c)\\
\textbf{where}\; f(x) = x(x+a)\\
\textbf{and}\; b = u/(u+l)\\
\textbf{and}\; c = v/(v-t-1)
\end{gather*}

In the expression above it is very natural to follow the grammar of the equation and deduce what is
happening. The where clause is naturally extended by adding \textbf{and}, which we can follow and logically
determine that this is an addition to the previous statement.

This is the type of grammatical reliance we must aim to avoid in the design of our language. We cannot
assume the end user of the language will be familiar with these grammatical primitives.

The design of ISWIM attempts to take this into consideration with multiple levels of abstraction in the
design of the language itself. The abstraction we are most interested in is what the author characterises as
logical ISWIM. This abstraction is defined as:

\say{uncommitted to character sets and type faces, but committed as to the sequence
of textual elements, and the grammatical rules for grouping them}

This is in theory would allow ISWIM to be translated to a different language and grammar completely
as long as the definition of ISWIM could be adapted to the end language's grammar. As it's definition is also
character set agnostic, it would naturally make sense that we could build a unique series of characters
which we assign meaning and grammar to in order to represent the language.

\section{Naturalistic Programming}

In \parencite{Lopes:2003:BAT:949344.949400} the authors outline a future beyond aspect-oriented
programming (AOP), a method of improving system modularity by modularizing crosscutting concerns \parencite{murphy}.
The authors propose that high level programming do not follow natural human communication patterns
or the way that we think. The next breakthrough will be by taking aspects of natural language
to make descriptions concise, effective and understandable.

There is a wide breadth of naturalistic type languages in use by programmers which are outlined in the survey
\parencite{Pulido-Prieto}. This is a fantastic survey of a wide array of high level
languages that follow grammatical and natural english rules to provide programmers with what is
ideally an easier understanding of code that is being written.

One such language which is among the most used languages in the world with multiple
dialects is Structured Query Language (SQL) \parencite{stackoverflow}. SQL is very expressive
and can be read like a regular English sentence. Without much knowledge of SQL at all you can make
deductions as to what the program will do by following its natural grammar rules.

Take the following expression for example:

\begin{verbatim}
    SELECT id, name, age, grade
    FROM students
    WHERE grade > 90
\end{verbatim}

Reading the expression we can deduce that this is accessing data by selecting various properties
\emph{from} some sort of students record \emph{where} the individuals grade is greater than 90.
Very natural to follow and conforms to some of Landin's ideas mentioned previously \parencite{Landin:1966:NPL:365230.365257}.

The programming language Pegasus is an example of the most complete naturalistic programming languages that
have been designed \parencite{Knoll:2006:PFS:1176617.1176628}.

The authors of this language highlight 4 major unsolved gaps in developer's expectations and programming techniques.

\begin{enumerate}
\item{Mental problem. An idea must be adapted to a programming language, decomposing orgrouping its elements.}

\item{Programming language problem. An algorithm must be translated into several languages that use new technologies and concepts, meaning that old languages still have a strong presence.}

\item{Natural language problem. In the modern era, while people from all around the worldwork in teams, they generally have a different native language and their own particular regionalisms.}

\item{Technical problem. In system design, the developer’s contribution and time is in-vested by programmers thinking about how to best adapt ideas to produce an efficient implementation.}
\end{enumerate}

We should pay particular attention to point number 3. The natural language and culture problem. With teams spanning the globe and working in increasingly more diverse and integrated environments
this problem becomes more prevalent. Even things as simple as color vs colour throw off developers every day when working with CSS, a markup that distinctly american.

Pegasus actually tackles this problem head on and is able to generate programs in English and German languages.
Using the definition of Pegasus proposed by the authors there has also been an extension to the language
Ara Pegasus.

\clearpage
\printbibliography

\end{document}
~
